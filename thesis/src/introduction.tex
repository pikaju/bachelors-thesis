\section{Introduction}
The field of \textit{artificial intelligence} as been extremely popular over recent years, due to the wide range of applications the technology has developed with the help of modern computer hardware. \textit{Machine learning algorithms} are capable of outperforming humans at a variety of different tasks, while being faster and significantly cheaper to operate. Possible applications for these algorithms include self-driving vehicles, computer-aided interpretation of medical imagery, stock market analysis and robotics. However, despite their impeccable performance, current artificially intelligent programs have a common drawback, which is their inability to generalize to other tasks. Each algorithm is developed by human experts specifically to solve a single problem and is hence referred to as \textit{artificial narrow intelligence}. A program capable of learning many different tasks, similarly to a human, is called \textit{artificial general intelligence} (\textit{AGI}), and developing such a program is the holy grail of artificial intelligence research.

Learning systems can be subdivided into three categories, which are \textit{supervised learning}, \textit{reinforcement learning} and \textit{unsupervised learning}. In supervised learning, the learner is presented with examples of inputs together with the desired outputs, and must adopt a mapping that generalizes to previously unseen inputs, providing useful outputs. For some problems, no examples that include their respective solutions exist, preventing the use of supervised learning. Consider the movement of a bipedal robot. Giving a sufficient amount of training examples on how to walk for a variety of different scenarios to a learner would require us to already have a solution to the problem we are trying to solve. It is, however, relatively trivial to judge the performance of said walker. For example, falling over is easy to detect and clearly undesirable, whereas a forwards movement should be encouraged. Learning only based on this feedback is called reinforcement learning. Finally, unsupervised learning is used for data in which neither a perfect solution nor a rating (as in the former categories) is available. These algorithms try to structure data by finding hidden patterns or similarities, which may, for instance, be used for visualization of the data.
