\newcommand{\policy}{\text{\textbf{p}}}
\newcommand{\svalue}{\mathfrak{v}}

\subsection{MuZero Algorithm}
The MuZero algorithm \cite{muzero} is a model-based reinforcement learning algorithm that builds upon the success of its predecessor, \textit{AlphaZero} \cite{alphazero}. Similar to other model-based algorithms, MuZero can predict the behavior of its environment to plan and choose the most promising action at each timestep to achieve its goal. In contrast to AlphaZero, it does this using a learned model. As such, it can be applied to environments for which the rules are not known ahead of time.

MuZero uses an \textit{internal} (or \textit{embedded}) state representation that is deduced from the environment observation but is not required to have any semantic meaning beyond containing sufficient information to predict rewards and values. Accordingly, it may be infeasible to reconstruct observations from internal states. This gives MuZero an advantage over algorithms akin to what is described in figure \ref{fig:recursive_model}. We explain this advantage using an example. Consider a robot agent receiving visual information through a camera as its observations. Predicting the future color of each of the potentially millions of pixels would be unnecessarily difficult and slow, given that we likely only need some key information, such as the position of an object in the environment. Now consider an alternative approach in which we first extract only key information from the observation that is relevant to our task, and then advance this derived knowledge into the future to make decisions. We can see that the latter approach may be advantageous.

There are three distinct functions (e.g. neural networks) that are used harmoniously for planning. Namely, there is a \textit{representation} function $h_\theta$, a \textit{prediction} function $f_\theta$, as well as a \textit{dynamics} function $g_\theta$, each being parameterized using $\theta$ to allow for adjustment through a training process. The complete model is called $\mu_\theta$. We will now explain each of the three functions in more detail. Note that we will diverge from our naming conventions to stay consistent with those employed by the MuZero paper.
\begin{itemize}
    \item The representation function $h_\theta$ is a mapping from real observations to internal state representations. For a sequence of recorded observations $o_1, \dots, o_t$ at timestep $t$, an embedded representation $s^0 = h(o_1, \dots, o_t)$ may be produced. As previously mentioned, $s^0$ has no semantic meaning, and typically contains significantly less information than $o_1, \dots, o_t$. Thus, the function $h_\theta$ is tasked with eliminating unnecessary details from observations, for example by extracting object coordinates and other attributes from images.

    \item The dynamics function $g_\theta$ tries to mimic the environment by advancing an internal state $s^{k-1}$ at a hypothetical timestep $k-1$ based on a chosen action $a^k$ to predict $r^k, s^k = g_\theta\left(s^{k-1}, a^k\right)$, where $r^k$ and $s^k$ are the reward and internal state at timestep $k$, respectively. This function can be applied recursively, similarly to what is shown in figure \ref{fig:recursive_model}, and acts as the simulating part of the algorithm, estimating what may happen when taking a sequence of actions $a^1, ..., a^k$ in a state $s^0$.

    \item The prediction function $f_\theta$ can be compared to an actor-critic architecture, having both a policy and a value output. For any internal state $s^k$, there shall be a mapping $\policy^k, v^k = f_\theta(s^k)$, where policy $\policy^k$ represents the probabilities with which the agent would perform actions and value $v^k$ is the expected return in state $s^k$. Whereas the value is very useful to bootstrap future rewards after the final step of planning by the dynamics function, the prediction function's policy may be counterintuitive, keeping in mind that the agent should derive its policy from the considered plans. We will explore the uses of $\policy^k$ further on.
\end{itemize}

Given parameters $\theta$, we can now decide on an action policy $\pi$ for each observation $o_t$, which we will call the \textit{search policy}, that uses $h_\theta$, $g_\theta$ and $f_\theta$ to search through different action sequences and find an optimal plan. As an example, in a small action space, we can iterate through all action sequences $a_1, ..., a_n$ of a fixed length $n$, and apply each function in the order visualized through figure \ref{fig:muzero_basic_policy}. By appropriately discounting the reward and value outputs with a discount factor $\gamma \in [0, 1]$, we receive an estimate for the return when first performing the action sequence and subsequently following $\pi$:
\begin{equation*}
    \mathbb{E}_\pi\left[
        G_t \given S_t \sim o_t, A_{t} = a_1, \dots, A_{t+n-1} = a_n
    \right] =
    \sum_{k=1}^n \gamma^{k-1} r^k + \gamma^n v^n
\end{equation*}
Given the goal of an agent to maximize the return, we may now simply choose the first action of the action sequence with the highest return estimate. Alternatively, a less promising action can be selected to encourage the exploration of new behavior. At timestep $t$, we call the return estimate for the chosen action produced by our search $\svalue_t$.
\begin{figure}[ht]
    \centering
    \begin{tikzpicture}[node distance=0.75]
        \node (ot) {$o_t$};
        \node [left = of ot] (otm1) {$o_{t-1}$};
        \node [right = of ot] (otp1) {$o_{t+1}$};
        \node [left = of otm1] (otm2) {$\dots$};
        \node [right = of otp1] (otp2) {$\dots$};
        \draw [->, dashed] (otm2) -- (otm1);
        \draw [->, dashed] (otm1) -- (ot);
        \draw [->, dashed] (ot) -- (otp1);
        \draw [->, dashed] (otp1) -- (otp2);

        \node [below = of ot, draw] (h) {$h_\theta$};
        \node [below = of h] (s0) {$s^0$};
        \draw [->] (ot) -- (h);
        \draw [->] (h) -- (s0);

        \node [right = of s0, draw] (g0) {$g_\theta$};
        \node [right = of g0] (s1) {$s^1$};
        \node [below left = of g0] (a1) {$a^1$};
        \node [above right = of g0] (r1) {$r^1$};
        \draw [->] (s0) -- (g0);
        \draw [->] (a1) edge [bend left] (g0);
        \draw [->] (g0) -- (s1);
        \draw [->] (g0) edge [bend right] (r1);

        \node [right = of s1, draw] (g1) {$g_\theta$};
        \node [right = of g1] (s2) {$s^2$};
        \node [below left = of g1] (a2) {$a^2$};
        \node [above right = of g1] (r2) {$r^2$};
        \draw [->] (s1) -- (g1);
        \draw [->] (a2) edge [bend left] (g1);
        \draw [->] (g1) -- (s2);
        \draw [->] (g1) edge [bend right] (r2);

        \node [right = of s2, draw] (g2) {$g_\theta$};
        \node [right = of g2] (s3) {$s^3$};
        \node [below left = of g2] (a3) {$a^3$};
        \node [above right = of g2] (r3) {$r^3$};
        \draw [->] (s2) -- (g2);
        \draw [->] (a3) edge [bend left] (g2);
        \draw [->] (g2) -- (s3);
        \draw [->] (g2) edge [bend right] (r3);

        \node [below = of s3, draw] (f) {$f_\theta$};
        \node [below = of f] (pv) {$\policy^3, v^3$};
        \draw [->] (s3) -- (f);
        \draw [->] (f) -- (pv);
    \end{tikzpicture}
    \caption{Testing a single action sequence made up of three actions $a^1$, $a^2$, and $a^3$ to plan based on observation $o_t$ using all three MuZero functions.}
    \label{fig:muzero_basic_policy}
\end{figure}

Training occurs on trajectories previously observed by the MuZero agent. For each timestep $t$ in the trajectory, we unroll our functions $K$ steps through time and adjust $\theta$ such that the predictions further match what was already observed in the trajectory. Each reward output $r^k_t$ of $g_\theta$ is trained on the real reward $u_{t+k}$ through a loss term $l^r$. We also apply the prediction function $f_\theta$ to each of the internal states $s^0_t, ..., s^K_t$, giving us policy predictions $\policy^0_t, ..., \policy^K_t$ and value predictions $v^0_t, ..., v^K_t$. Each policy prediction $\policy^k_t$ is trained on the stored search policy $\pi_{t+k}$ with a policy loss $l^p$. This makes $\policy$ an estimate (that is faster to compute) of what our search policy $\pi$ might be, meaning it can be used as a heuristic. For our value estimates $\svalue$, we first calculate $n$-step bootstrapped returns $z_t = u_{t+1} + \gamma u_{t+2} + ... + \gamma^{n-1} u_{t+n} + \gamma^n\svalue_{t+n}$. With another loss $l^v$, the target $z_{t+k}$ is set for each value output $v^k_t$. The three previously mentioned output targets as well as an additional \textit{L2 regularization} \cite{l2-regularization} term $c||\theta||^2$ form the complete loss function:
\begin{equation*}
    l_t(\theta) = \sum^K_{k=0} \left(l^r(u_{t+k}, r_t^k) + l^v(z_{t+k}, v_t^k) + l^p(\pi_{t+k}, \policy_t^k) + c||\theta||^2\right)
\end{equation*}

The aforementioned brute-force search policy is highly unoptimized. For one, we have not described a method of caching and reusing internal states and prediction function outputs so as to reduce the computational footprint. Furthermore, iterating through all possible actions is very slow for high numbers of actions or longer action sequences and impossible for continuous action spaces. Ideally, we would want to focus our planning efforts on only those actions that are promising from the beginning and spend less time incorporating clearly unfavorable behavior. AlphaZero and MuZero employ a variation of \textit{Monte-Carlo tree search} (\textit{MCTS}) \cite{mcts}, which we will elaborate in the following section.

It is advisable to use a replay buffer for MuZero to reach its potential with regards to sample efficiency. That is, we store trajectories gathered through interaction with the environment in a buffer for some time and reuse the same experience samples multiple times during training. Because the losses are calculated based on the respective search policy $\pi$ and value $\svalue$, $\pi$ and $\svalue$ must be readily available. Recalculating search results for each training iteration would take a tremendous amount of time and processing power, which is why we instead store them alongside the actual experience after they are determined during environment interaction. This, in turn, means that the stored policies and values, produced by the parameterized model $\mu_\theta$, become increasingly outdated as the training progresses and $\theta$ is updated. Nevertheless, empirical evidence shows that the algorithm works despite this inaccuracy. However, the problem can be mitigated through an advanced method called \textit{MuZero Reanalyze} \cite{muzero}, in which replay samples are regularly updated by performing the search algorithm with the latest available model parameters. This has been shown to further improve MuZero's performance but comes at the cost of significantly larger computational requirements.

MuZero matches the state-of-the-art results of AlphaZero in the board games Chess and \textit{Shogi}, despite not having access to a perfect environment model. It even exceeded AlphaZero's unprecedented rating in the board game \textit{Go}, while using less computation per node, suggesting that it caches useful information in its internal states to gain a deeper understanding of the environment with each application of the dynamics function. Furthermore, MuZero outperforms humans and previous state-of-the-art agents at the Atari game benchmark, demonstrating its ability to solve tasks with long time horizons that are not turn-based.