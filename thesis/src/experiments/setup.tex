\subsection{Setup}
We want to test our hypothesis suggesting a performance increase when applying the abovementioned changes to the MuZero agent. Because MuZero is a specialization of . Performance is measured by training an agent for a specific amount of training steps and, at various time steps, sampling the total episode reward the agent can achieve.

We use \textit{muzero-general} \cite{muzero-general}, an open source implementation of MuZero that uses the Python programming language and PyTorch for automatic differentiation, as our baselines agent, and modify the code to include a reconstruction function and additional loss terms. The readout of the replay buffer must also be tweaked to include not only the observation at timestep $t$, but the $K$ subsequent observations $o_{t+1}, ..., o_{t+K}$ as well, as they are critical for calculating our new loss values.

Table \ref{tab:hyperparameters} shows the hyperparameters used by all tested model configurations. Most of the listed hyperparameters were kept as is from the muzero-general defaults. Note that we are unable to perform a comprehensive hyperparameter search due to technical limitations.
\begin{table}[ht]
    \centering
    \begin{tabular}{|l||c|c|}
        \hline
        & CartPole-v1 & LunarLander-v2 \\
        \hline\hline
        Simulations & 50 & 50 \\
        \hline
        Discount factor ($\gamma$) & 0.997 & 0.999 \\
        \hline
        Support size\footnote{Number of dimensions in the internal state vectors} & 10 & 10 \\
        \hline
        Training steps & 10000 & 30000 \\
        \hline
        Batch size & 128 & 64 \\
        \hline
        Value loss weight & 1.0 & 1.0 \\
        \hline
        Optimizer & Adam & Adam \\
        \hline
        L2 regularization weight & $10^{-4}$ & $10^{-4}$ \\
        \hline
        Learning rate ($\beta$) & $0.02 \times 0.9^{t \times 0.001}$ & 0.005 \\
        \hline
        Replay buffer size & 500 & 2000 \\
        \hline
        Unroll steps ($K$) & 10 & 10 \\
        \hline
        TD steps ($n$) & 50 & 30 \\
        \hline
        Replay prioritization exponent & 0.5 & 0.5 \\
        \hline
        Softmax temperature ($T$) & \makecell{
            1.0 if $t<5000$, \\ 0.5 if $5000 \leq t < 7500$, \\ 0.25 if $t \geq 7500$
        } & 0.35 \\
        \hline
    \end{tabular}
    \caption{Hyperparameter selection for all tested agents. Parameters based on the current training step use the variable $t$.}
    \label{tab:hyperparameters}
\end{table}
